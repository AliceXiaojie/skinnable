\documentclass{chi-ext}

\newcommand{\papertitle}{Towards ``Skinnable'' Physical Objects}
\title{\papertitle}

\numberofauthors{3}
% Notice how author names are alternately typesetted to appear ordered
% in 2-column format; i.e., the first 4 autors on the first column and
% the other 4 auhors on the second column.  Actually, it's up to you
% to strictly adhere to this author notation.
\author
{
  \alignauthor
	{
  	\textbf{First Author}\\
  	\affaddr{AuthorCo, Inc.}\\
  	\affaddr{123 Author Ave.}\\
  	\affaddr{Authortown, PA 54321 USA}\\
  	\email{author1@anotherco.com}
  }
  \vfil
  \alignauthor
	{
  	\textbf{Second Author}\\
  	\affaddr{AuthorCo, Inc.}\\
  	\affaddr{123 Author Ave.}\\
  	\affaddr{Authortown, PA 54321 USA}\\
  	\email{author2@anotherco.com}
  }
  \vfil
  \alignauthor
	{
  	\textbf{Third Author}\\
  	\affaddr{AuthorCo, Inc.}\\
  	\affaddr{123 Author Ave.}\\
  	\affaddr{Authortown, PA 54321 USA}\\
  	\email{author3@anotherco.com}
  }
}

% Paper metadata (use plain text, for PDF inclusion and later
% re-using, if desired)
\hypersetup
{
  % Your metadata go here
  pdftitle={\papertitle},
  pdfauthor={authors},  
  pdfkeywords={},
  pdfsubject={},
  % Quick access to color overriding:
  citecolor=black,
  linkcolor=black,
  menucolor=black,
  urlcolor=black,
}

\usepackage{graphicx}   % for EPS use the graphics package instead
\usepackage{balance}    % useful for balancing the last columns
\usepackage{bibspacing} % save vertical space in references


\begin{document}

\maketitle

\begin{abstract}
	In this sample we describe the formatting requirements for various SIGCHI related submissions 
	and offer recommendations on writing for the worldwide SIGCHI readership. 
	%Do not change the page size or page settings.
	Please review this document even if you have submitted to SIGCHI conferences before, 
	some format details have changed relative to previous years.
\end{abstract}

\keywords{}
\textcolor{red}{Optional section to be included in your final version.}

\category{H.5.m}{Information interfaces and presentation (e.g., HCI)}{Miscellaneous}. 
%See \cite{ACMCCS} 
See: \url{http://www.acm.org/about/class/1998/} 
for help using the ACM Classification system.
\textcolor{red}{Optional section to be included in your final version, but strongly encouraged.}


\section{Introduction}

	Motivation: we want to let non-technical people download
	pre-configured component-based-assembly circuits (e.g. LittleBits)
	and print out cases for them. For example, the Twitter bird. Or the
	weather statue.


\section{Related Work}

	With the emergence of digital fabrication technologies, there have
	been a number of projects intended to simplify the process of
	creating interactive objects.

	Two recent projects considered building customized enclosures for
	exiting electronics. Weichel et~al.\ created a system for designing
	laser-cut cases for projects created with the .NET
	Gadgeteer electronics prototyping platform \cite{Weichel:2013kn}.

	Building enclosures:
		\cite{Schneegass:2014ip} (NatCut: An Interactive Tangible Editor for
			Physical Object Fabrication)
		\cite{Weichel:2013kn} (Enclosed: a component-centric interface for
			designing prototype enclosures)

	
\section{Stuff}
	Features:
	\begin{itemize}
		\item Represent the circuit as an undirected graph (currently
			input by hand in code).
		\item 2D packing to find the best arrangement.
		\item Minimize cost function of size, number of wires; something
			like $c = wS + (1-w)W$ where $S$ is size, $W$ is number of
			wires, and $w$ is relative weight between the two.
		\item Specify location of fixed modules (in code).
		\item Search all possible logical arrangments of wires/not wires.
		\item Output object for printing.
	\end{itemize}




\balance
\bibliographystyle{acm-sigchi}
\bibliography{sample}

\end{document}
